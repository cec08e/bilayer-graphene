\documentclass{article}
\usepackage{amsthm}

\title{The Tight Binding Model}
\author{Caitlin Carnahan}
\begin{document}
\maketitle
\begin{abstract}
This document is intended as a detailed review of covalent bonding and the Tight Binding (LCAO) model in preparation for study of monolayer and bilayer graphene.
\end{abstract}
\section{Covalent Bonding}
Graphene is composed of hexagonally-shaped links of carbon atoms, each of which is linked to three other carbon atoms when the hexagons are tiled. These carbon atoms are linked together by covalent bonds.
The purpose of this section is to review the fundamental concepts of covalent bonding, as well as the properties of the carbon atom itself. \par
Carbon is the chemical element with atomic number 6 - that is, there are six protons found in the nucleus of a Carbon atom. The ground-state electron configuration of Carbon is $1s^{2}2s^{2}2p^{2}$. However, when in the presence of other atoms,
it is sometimes energetically favorable to promote one electron into the $2p$ orbital - the energy cost of moving into this excited state is offset by the energy gained from covalent bonding. \par
In the case of graphene, carbon atoms exibit $sp^{2}$ hybridization wherein two $2p$ orbitals, conventionally chosen as $2p_{x}$ and $2p_{y}$, form a superposition with the $2s$ orbital. The result is three coplanar orbitals separated by $120^{\circ}$.
The remaining $2p_{z}$ orbital lies perpendicular to the coplanar hybridized orbitals. Sheets of graphene are formed by planes of carbon atoms bonding via the hybridized orbitals, but this leaves the perpendicular $2p_{z}$ orbital free. \par
We now turn our attention to the nature of covalent bonding in graphene. What follows is a description of covalent bonding that closely follows that presented in \cite{oxford}. As a simple model, we can demonstrate the energetically favorable effects
produced by covalent bonding if we imagine an atom as a potential well for an electron - specifically, we model an atom as an infinite square well. The ground state solution for a single electron in the infinite square well is given by: $$E = \frac{\pi^{2}\hbar^{2}}{2mL^{2}}$$





\begin{thebibliography}{9}
\bibitem{oxford}
Steven H. Simon.
\textit{Lecture Notes for Solid State Physics}.
Oxford University, 2012.

\end{thebibliography}



\end{document}

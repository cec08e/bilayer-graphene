\documentclass{article}
\usepackage{amsthm}
\usepackage{graphicx}
\usepackage{subfig}
\usepackage{physics}
\graphicspath{ {figures/} }

\title{Topological Confinement in Bilayer Graphene}

\begin{document}
\section{Effective Hamiltonian}
We begin with the low-energy bilayer Hamiltonian (in the $\psi_{A_{1}}$, $\psi_{B_{1}}$, $\psi_{A_{2}}$, $\psi_{B_{2}}$ basis).

\begin{equation}
H =
\begin{bmatrix}
0 & \frac{\sqrt{3}a}{2}\kappa\gamma_{1} & 0 & \frac{\sqrt{3}a}{2}\kappa^{*}\beta_{nn}\\
\frac{\sqrt{3}a}{2}\kappa^{*}\gamma_{1} & 0 & \beta_{d} & 0 \\
0 & \beta_{d} & 0  & \frac{\sqrt{3}a}{2}\kappa\gamma_{1} \\
\frac{\sqrt{3}a}{2}\kappa\beta_{nn} & 0 & \frac{\sqrt{3}a}{2}\kappa^{*}\gamma_{1}& 0
\end{bmatrix}
\end{equation}

We will further allow the $\beta_{nn}$ coupling terms to vanish and add an additional bias voltage term to the Hamiltonian.

\begin{equation}
H =
\begin{bmatrix}
0 & \frac{\sqrt{3}a}{2}\kappa\gamma_{1} & 0 & 0\\
\frac{\sqrt{3}a}{2}\kappa^{*}\gamma_{1} & 0 & \beta_{d} & 0 \\
0 & \beta_{d} & 0  & \frac{\sqrt{3}a}{2}\kappa\gamma_{1} \\
0 & 0 & \frac{\sqrt{3}a}{2}\kappa^{*}\gamma_{1}& 0
\end{bmatrix}
 + h_{bias}
\end{equation}

where the modification to the Hamiltonian due to an interlayer bias voltage is given by

\begin{equation}
h_{bias} =
\begin{bmatrix}
-\frac{V(x)}{2} & 0 & 0 & 0\\
0& -\frac{V(x)}{2} & 0 & 0 \\
0 & 0 & \frac{V(x)}{2} & 0 \\
0 & 0 &  0 & \frac{V(x)}{2} \\
\end{bmatrix}
\end{equation}

If we choose to let $\pi = \kappa^{\dagger} = p_{x} + ip_{y}$ (for notational reasons), $\beta_{d} = t_{\perp}$, and the Fermi
velocity $c = \frac{\sqrt{3}a}{2}\gamma_{1}$, then we have for our full Hamiltonian:

\begin{equation}
H =
\begin{bmatrix}
-\frac{V(x)}{2}  & c\pi^{\dagger} & 0 & 0\\
c\pi & -\frac{V(x)}{2}  & t_{\perp} & 0 \\
0 & t_{\perp} & \frac{V(x)}{2}  & c\pi^{\dagger} \\
0 & 0 & c\pi & \frac{V(x)}{2}
\end{bmatrix}
\end{equation}


To obtain a nicer form for the effective Hamiltonian,
we begin by reorganizing the basis: $\{A_{1}, B_{1}, A_{2}, B_{2}\} \rightarrow \{A_{1}, B_{2}, A_{2}, B_{1}\}$. We obtain

\begin{equation}
H =
\begin{bmatrix}
-\frac{V(x)}{2} & 0 & 0 & c\pi^{\dagger}\\
0& \frac{V(x)}{2}  & c\pi & 0 \\
0 & c\pi^{\dagger}  & \frac{V(x)}{2}  & t_{\perp} \\
c\pi & 0 & t_{\perp} & -\frac{V(x)}{2}
\end{bmatrix} =
\begin{bmatrix}
H_{11} & H_{12} \\
H_{21} & H_{22}
\end{bmatrix}
\end{equation}
We make use of the identity
\begin{equation}
det(H - E) = det(H_{11} - H_{12}(H_{22} - E)^{-1}H_{21} - E)det(H_{22} - E)
\end{equation}

Here, we will assume that $V << t_{\perp}$ and since we are concerned with the
low energy region where $E << t_{\perp}$, we can make the approximation
$$
H_{22} - E \approx H_{22}^{'} =
\begin{bmatrix}
0 & t_{\perp}\\
t_{\perp} & 0
\end{bmatrix}
$$
This approximation allows us to write
$$
det(H_{22} - E) = -t_{\perp}^{2}
$$
and the effective $2\times2$ Hamiltonian is
\begin{equation}
H_{eff} = \left ( H_{11} - H_{12}H_{22}^{-1}H_{21} \right )
\end{equation}
We will make use of the fact that $V(x) << t_{\perp}$ to simplify $H_{22}^{-1}$:
$$
H_{22}^{-1} = \frac{1}{-(\frac{V(x)}{2})^2 - t_{\perp}^{2}}
\begin{bmatrix}
\frac{-V(x)}{2} & -t_{\perp}\\
-t_{\perp} & \frac{V(x)}{2}
\end{bmatrix}
\approx
\begin{bmatrix}
\frac{V(x)}{2t_{\perp}^{2}} & \frac{1}{t_{\perp}}\\
\frac{1}{t_{\perp}} & \frac{-V(x)}{2t_{\perp}^{2}}
\end{bmatrix}
$$
That is,
\begin{equation}
H_{eff} =
\begin{bmatrix}
\frac{-V(x)}{2} & 0\\
0 & \frac{V(x)}{2}
\end{bmatrix}
-
\begin{bmatrix}
0 & c\pi^{\dagger}\\
c\pi & 0
\end{bmatrix}
\begin{bmatrix}
\frac{V(x)}{2t_{\perp}^{2}} & \frac{1}{t_{\perp}}\\
\frac{1}{t_{\perp}} & \frac{-V(x)}{2t_{\perp}^{2}}
\end{bmatrix}
\begin{bmatrix}
0 & c\pi^{\dagger}\\
c\pi & 0
\end{bmatrix}
\end{equation}
$$
=
\begin{bmatrix}
\frac{-V(x)}{2} & 0\\
0 & \frac{V(x)}{2}
\end{bmatrix}
-
\begin{bmatrix}
\frac{-V(x)c^{2}|\pi|^{2}}{2t_{\perp}^2} & \frac{(c\pi^{\dagger})^{2}}{t_{\perp}}\\
\frac{(c\pi)^{2}}{t_{\perp}} & \frac{V(x)c^{2}|\pi|^{2}}{2t_{\perp}^2}
\end{bmatrix}
$$
We finally obtain
\begin{equation}
H_{eff} =
\begin{bmatrix}
\frac{-V(x)}{2}(1 - \frac{c^{2}|\pi|^{2}}{t_{\perp}^2}) & -\frac{(c\pi^{\dagger})^{2}}{t_{\perp}}\\
-\frac{(c\pi)^{2}}{t_{\perp}} & \frac{V(x)}{2}(1 - \frac{c^{2}|\pi|^{2}}{t_{\perp}^2})
\end{bmatrix}
\end{equation}
The basis of which is the non-dimer sites ${A_{1}, B_{2}}$.
Diagonalizing the effective Hamiltonian, we can now see that the energies are given by
\begin{equation}
E^{2} = \frac{(c|\pi|)^{4}}{t_{\perp}^{2}} + \frac{V(x)^{2}}{4}\left (1 - \frac{c^{2}|\pi|^{2}}{t_{\perp}^2}\right )^{2}
\end{equation}
But if V(x) is significantly less than the hopping parameter $t_{\perp}$, then we can simplify the effective Hamiltonian and corresponding spectrum as
\begin{equation}
H_{eff} =
\begin{bmatrix}
\frac{-V(x)}{2} & -\frac{(c\pi^{\dagger})^{2}}{t_{\perp}}\\
-\frac{(c\pi)^{2}}{t_{\perp}} & \frac{V(x)}{2}
\end{bmatrix}
\end{equation}
\begin{equation}
E^{2} = \frac{(c|\pi|)^{4}}{t_{\perp}^{2}} + \frac{V(x)^{2}}{4}
\end{equation}
In order to rewrite the effective Hamiltonian in a quasiclassical form, we begin by factoring out
$\frac{c^2}{t_{\perp}}$
\begin{equation}
H_{eff} =
\begin{bmatrix}
\frac{-V(x)t_{\perp}}{2c^{2}} & -(\pi^{\dagger})^{2}\\
-\pi^{2} & \frac{V(x)t_{\perp}}{2c^{2}}
\end{bmatrix}
\end{equation}
\begin{equation}
=
\begin{bmatrix}
\frac{-V(x)t_{\perp}}{2c^{2}} & -(p_{x}^{2} - p_{y}^{2} - 2ip_{x}p_{y})\\
-(p_{x}^{2} - p_{y}^{2} + 2ip_{x}p_{y}) & \frac{V(x)t_{\perp}}{2c^{2}}
\end{bmatrix}
\end{equation}
Finally, if we define the momenta to be measured in units of $1/a$, then we finally obtain
\begin{equation}
H_{qc} = -\phi(x)\sigma_{z} - (p_{x}^{2} - p_{y}^{2})\sigma_{x} - 2p_{x}p_{y}\sigma_{y}
\end{equation}
where we define
$$
\phi(x) = \frac{V(x)t_{\perp}a^{2}}{2c^{2}}
$$
We can also further define a function $\vec{g}(\vec{p}, x)$ such that
\begin{equation}
H_{qc} = \vec{g}(\vec{p}, x)\cdot\vec{\sigma}
\end{equation}
The corresponding wave equation is given by
\begin{equation}
H(\vec{p}, x)\psi = \epsilon\psi
\end{equation}
For $\psi = [u(x), v(x)]$, we obtain
$$
-\phi(x)u(x) + (ip_{x} + p_{y})^{2}v(x) = \epsilon u(x)
$$
\begin{equation}
\phi(x)v(x) + (ip_{x} - p_{y})^{2}u(x) = \epsilon v(x)
\end{equation}
And using $\partial_{x} = -ip_{x}$ in natural units where $\hbar = 1$,
$$
-\phi(x)u(x) + (\partial_{x} + p_{y})^{2}v(x) = \epsilon u(x)
$$
\begin{equation}
\phi(x)v(x) + (\partial_{x} - p_{y})^{2}u(x) = \epsilon v(x)
\end{equation}
If we enforce an antisymmetric potential profile (i.e. $\phi(-x) = -\phi(x)$), then,
$$
-\phi(x)u(x) + (\partial_{x} + p_{y})^{2}v(x) = \epsilon u(x)
$$
\begin{equation}
-\phi(x)v(-x) + (\partial_{x} + p_{y})^{2}u(-x) = \epsilon v(-x)
\end{equation}
and we can see that if $v(x) = \pm u(-x)$, then we have two systems of equations given by:
$$
-\phi(x)u(x) + (\partial_{x} + p_{y})^{2}u(-x) = \epsilon u(x)
$$
$$
-\phi(x)u(x) + (\partial_{x} + p_{y})^{2}u(-x) = \epsilon u(x)
$$
and
$$
-\phi(x)w(x) + (\partial_{x} + p_{y})^{2}(-w(-x)) = \epsilon w(x)
$$
$$
-\phi(x)(-w(x)) + (\partial_{x} + p_{y})^{2}(w(-x)) = -\epsilon w(x)
$$
But all of these equations can be solved in the same way, as they are just the same equation. So, for a particular value
of $p_{y}$, we can obtain an eigenvector $\Psi_{p_{y}} = [u_{p_{y}}(x), u_{p_{y}}(-x)]$ with eigenvalue $\epsilon_{p_{y}}$
that solves the first set of equations.
%That is,
%$$
%-\phi(x)u(x) + (\partial_{x} + p_{y})^{2}u(-x) = \epsilon u(x)
%$$
We can also obtain a second eigenvector $\Phi_{p_{y}} = [w_{p_{y}}(x), -w_{p_{y}}(-x)]$ that solves the second set of
equations. From the first eigenvector solution, we have:
$$
-\phi(x)u(x) + (\partial_{x} + p_{y})^{2}u(-x) = \epsilon u(x)
$$
$$
-\phi(x)u(x) + (\partial_{x} + p_{y})^{2}u(-x) = \epsilon u(x)
$$
Inverting the sign on x gives us
$$
-\phi(-x)u(-x) + (-\partial_{x} + p_{y})^{2}u(x) = \epsilon u(-x)
$$
$$
-\phi(-x)u(-x) + (-\partial_{x} + p_{y})^{2}u(x) = -\epsilon(-u(-x))
$$
And using the antisymmetry of the potential profile:
$$
-\phi(x)(-u(-x)) + (-\partial_{x} + p_{y})^{2}u(x) = \epsilon u(-x)
$$
$$
-\phi(x)(-u(-x)) + (-\partial_{x} + p_{y})^{2}u(x) = -\epsilon(-u(-x))
$$
We know the solutions to these equations are $[]$
If we swap the sign of $x$, we obtain the equivalent result
$$
-\phi(-x)u(-x) + (\partial_{x} - p_{y})^{2}u(x) = \epsilon u(-x)
$$
And making use of the antisymmetry of the potential profile, we can write this as
$$
-\phi(x)(-u(-x)) + (\partial_{x} - p_{y})^{2}u(x) = -\epsilon(-u(-x))
$$
So, we therefore obtain another solution $\Phi_{p_{y}} = [-u_{-p_{y}}(-x), u_{-p_{y}}(x)]$ with
eigenvalue $-\epsilon_{-p_{y}}$. Our goal now is to analyze the dispersion $\epsilon(p_{y})$ to
observe the effect of the form of $\phi(x)$.

\section{Step Kink}
We will consider a step-like kink $\phi(x) = \phi_{0}sgn(x)$, where
\begin{equation}
sgn(x) = \left\{\begin{matrix}
 1& x > 0\\
 0& x = 0\\
 -1 & x < 0
\end{matrix}\right.
\end{equation}

Our wave equation becomes
\begin{equation}
\begin{matrix}
-u(x) + (\partial_{x} + p_{y})^{2}v(x) = \epsilon u(x) & x > 0\\
u(x) + (\partial_{x} + p_{y})^{2}v(x) = \epsilon u(x) & x < 0
\end{matrix}
\end{equation}

For $\Psi = [u(x), v(x) = u(-x)]$, we have
\begin{equation}
\begin{matrix}
-\phi_{0}u(x) + (\partial_{x} + p_{y})^{2}u(-x) = \epsilon u(x) & x > 0\\
\phi_{0}u(x) + (\partial_{x} + p_{y})^{2}u(-x) = \epsilon u(x) & x < 0
\end{matrix}
\end{equation}
In the $x > 0$ region, we can rewrite this as
\begin{equation}
-\phi_{0}u(x) + (\partial_{x}^{2} + \partial_{x}p_{y} + p_{y}^{2})u(-x) = \epsilon u(x)
\end{equation}
where we use the operator algebra $\partial_{x}p_{y} = p_{}$

In both the $+x$ and $-x$ regions, the wave equation takes the form of a second order differential equation,
which has general solution $\Psi \propto e^{-\lambda x}$, where $\lambda$ can be a complex root.

\end{document}
